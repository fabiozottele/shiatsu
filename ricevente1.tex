\documentclass[12pt, openright, twoside]{report}
\usepackage[T1]{fontenc}
\usepackage[italian]{babel}
\usepackage{siunitx}
\DeclareSIUnit\bpm{bpm}
\usepackage{enumitem}
\newcommand{\RNum}[1]{\uppercase\expandafter{\romannumeral #1\relax}}
\newcommand{\yang}{Yang}

\begin{document}
\title{Tesi}
\author{Fabio Zottele}
\date{\today}
\maketitle
\begin{abstract}
spiegare periodi, riceventi e i vari perché e percome...
\end{abstract}
\chapter*{Ricevente n°1}
Tratterò un maschio italiano di 43 anni, alto \SI{1,75}{\centi\metre}
per \SI{75}{\kilogram}. \`{E} un programmatore informatico ed ha iniziato da un
anno un percorso di dottorato di ricerca: è un lavoro principlamente sedentario.
Non comunica particolari problemi fisici.

\`{E} sportivo e allenato: climber, motociclista pratica mountain bike, corre
mediamente \SI{20}{\kilo\metre} a settimana e si sta preparando per una gara
granfondo di sci competitiva, gara alla quale ha già partecipato due volte con
ottimi risultati.

Si definisce bradicardico (\SI{42}{\bpm} a riposo), fumatore regolare
e non comunica allergie e intolleranze alimentari.

Non ha mai ricevuto un trattamento shiatsu né massaggi di tipo fisioterapico.
\subsection*{Primo incontro: 21 ottobre 2022}
\subsubsection{Colloquio}
Il ricevente lamenta mal di schiena non localizzato, ma esteso a livello della
zona della \RNum{1} e \RNum{2} vertebra lombare, nella parte destra del corpo.
Muovendo la schiena in torsione il dolore si trasferisce anche nella parte
sinistra dell'area di diagnosi.
Cambiando posizione il dolore passa, ma non è chiaro se ci sia una posizione che
fa passare il dolore che \textit{è simile ad un arco che tira}, con sensazioni
pungenti.

Il ricevente lamenta dolore al collo la sera nelle giornate in cui utilizza il
calcolatore elettronico per molto tempo.
Il ricevente è sicuro che il dolore è legato al tipo di attività.
Risultano dolorosi tutti i movimenti del collo.
Il dolore passa nel momento in cui il ricevente si corica.
\`{E} un dolore di tipo \textit{fisso, pungente} e viene indicato dal ricevente
nella zona che va dalla \RNum{7} vertebra cervicale e salendo fino alla base del
capo e a scendere coinvolge le spalle: viene indicata la zona laterale del
collo, la zona del muscolo trapezio fino al tetto della spalla e poi - a
scendere - la parte superiore della scapola verso la spalla.

Per tutto il colloquio il ricevente siede in maniera composta ed eretta.
I movimenti degli arti sono appena accennati e molto composti.
Alle domande risponde in maniera precisa, con poche parole, senza aggiungere
dettagli.
Ha lo sguardo attento e fisso.

\subsubsection*{Anamnesi}
Il ricevente mi ha indicato due zone dolorose:
\begin{itemize}[font=\bfseries, align=left]
\item[Schiena:] il dolore (jitsu) si colloca nella parte Yin dell’area di diagnosi posteriore di Intestino Tenue. Tale dolore si muove con attitudine Yang invadendo la parte Yang dell’area di diagnosi.

\item[Collo:] l’indicazione data dal ricevente sembra coinvolgere l’area percorsa dai meridiani di Stomaco, Vescicola Biliare (nella parte più esterna del muscolo trapezio), Triplice Riscaldatore,  Intestino Tenue e Vescica. Potenzialmente sono coinvolti sia gli organi (Yin) che i visceri (\yang) collegati alle energie di Legno, Fuoco, Terra e Acqua. Il quadro indicativo è la tendenza all’instaurazione di un quadro jitsu di tipo Yin con tendenza all'auto risoluzione.
\end{itemize}
\chapter*{Ricevente}

\end{document}
